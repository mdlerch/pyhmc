\documentclass[10pt,letterpaper]{article}
\usepackage{fullpage}
\usepackage[]{amsmath}
\usepackage{amssymb}
\usepackage{subfig}
\usepackage{wrapfig}
\usepackage{float}
\usepackage{setspace}
\usepackage{graphicx}

\pagestyle{empty}

\newcommand{\Sum}{\displaystyle\sum}
\newcommand{\Int}{\displaystyle\int}
\newcommand{\Frac}{\displaystyle\frac}
\newcommand{\Not}[1]{\overline{#1}}

\setstretch{1.2}

\author{Michael Lerch}
\title{PyHMC}
\date{}

\begin{document}
\maketitle

\section{About}
\label{sec:about}

PyHMC is a tool that utilizes the Hamiltonian Monte Carlo technique for
Bayesian methods that samples from posterior distributions.  Currently

% section about (end)

\section{Usage}
\label{sec:usage}

\subsection{Demo mode}
\label{sub:demo_mode}

Demo mode can be set with the flag \texttt{--demo}.  The demo mode can be used
to sample from a bivariate normal distribution.

Example:

\begin{verbatim}
pyhmc --demo
\end{verbatim}

% subsection demo_mode (end)

\subsection{Sampling mode}
\label{sub:sampling_mode}

A likelihood needs to be selected to run in sampling mode.  Currently, two
likelihoods are supported: binomial and univariate normal.  To select between
these likelihoods the argument \texttt{binomial} or \texttt{normal} must be
provided to \texttt{pyhmc}.  Sampling mode samples from the posterior
distribution with the specified likelihood and a improper uniform prior on the
support of the canonical parameters.

Example:

\begin{verbatim}
pyhmc binomial # sample from a posterior distribution with binomial likelihood
pyhmc normal # sample from a posterior distribution with normal likelihood
\end{verbatim}

% subsection sampling_mode (end)

\subsection{Options}
\label{sub:options}

\texttt{pyhmc} takes many options to tune the sampling algorithm.  These
options include:

\begin{itemize}
	\item Number of samples \texttt{-S samples}.  The default value is 1000.  To set 10
samples with binomial likelihood:

\begin{verbatim}
	pyhmc binomial -S 10
\end{verbatim}

	\item Number of integrator steps \texttt{-L steps}.  The default value is
		10.  To set 5 integrator steps with binomial likelihood:

\begin{verbatim}
	pyhmc binomial -L 5
\end{verbatim}
	
	\item Time step between integrator steps \texttt{-E step\_size} or
		\texttt{--epsilon step\_size}.  The default value is
		0.1.  To set 0.05 time step between integrator steps:

\begin{verbatim}
	pyhmc binomial -E 0.05 # or pyhmc binomial --epsilon 0.05
\end{verbatim}

	\item Initial coordinates \texttt{-I q0 [q1]}.  The default value changes
		depending on the likelihood.  For binomial, input one value, the
		initial value for $\pi$.  For normal, input two values, the intial
		value for $\mu$ and the initial value for $\sigma$.  For demo mode,
		input two values the initial coordinates $y_1$ and $y_2$.  To set the
		initial value of $\pi$ to 0.6 with binomial likelihood:

\begin{verbatim}
	pyhmc binomial -I 0.6
\end{verbatim}

	\item Data \texttt{-Y data0 data1}.  This option is available for binomial
		sampling and demo mode.  For binomial, input the number of observed
		successes and total number of observations.  For demo mode, input the
		two values for the mean vector of the bivariate normal distribution.
		For binomial likelihood, setting 27 successes out of 50 observations as
		the data:

\begin{verbatim}
	pyhmc binomial -Y 27 50
\end{verbatim}

	\item Data file \texttt{--infile datafile}.  This option is available for
		binomial and normal sampling.  For binomial, the input file should be a
		single column of 1's and 0's.  For normal, the input file should be a
		single column of observations.  Each observation is on it's own row.
		This option overrides the command line data input.  For a binomial
		likelihood and data contained in \texttt{infilebinomial.txt}:

\begin{verbatim}
	pyhmc binomial --infile infilebinomial.txt
\end{verbatim}

	\item Output file \texttt{-O outfile}.  The default value is
		\texttt{output.txt}.  This sets the output file to record the samples.
		If the file currently exists, \texttt{pyhmc} will quit without
		overwriting the existing file.  For a binomial likelihood to store in
		\texttt{outfilebinomial.txt}

\begin{verbatim}
	pyhmc binomial -O outfilebinomial.txt
\end{verbatim}

	\item Be verbose \texttt{-V}.  After every step, show the current position
		and momentum (in canonical parameter space) and show the Metropolis
		Hastings $\alpha$ at sampling time.  For trouble shooting.
\end{itemize}

\section{Issues}
\label{sec:issues}

At present normal likelihood sampling is very unstable.  Results should not be
trusted.

% section issues (end)

% subsection options (end)

% section usage (end)


\end{document}
